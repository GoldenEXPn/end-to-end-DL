\documentclass[11pt, oneside]{article}   	% use "amsart" instead of "article" for AMSLaTeX format
\usepackage{geometry}                		% See geometry.pdf to learn the layout options. There are lots.
\geometry{letterpaper}                   		% ... or a4paper or a5paper or ... 
%\geometry{landscape}                		% Activate for rotated page geometry
\usepackage[parfill]{parskip}    			% Activate to begin paragraphs with an empty line rather than an indent
\usepackage{graphicx}				% Use pdf, png, jpg, or eps§ with pdflatex; use eps in DVI mode
								% TeX will automatically convert eps --> pdf in pdflatex		
\usepackage{amssymb}
\usepackage{hyperref}
\usepackage{mathtools}
\usepackage{enumerate}
\usepackage{tikz}
% \usepackage[htt]{hyphenat}
\usepackage[normalem]{ulem}

\newcommand{\ck}[1]{\textcolor{cyan}{CK: #1}}
\newcommand{\jc}[1]{\textcolor{blue}{JC: #1}}


\title{Homework 1 \\ CSC 277 / 477 \\ End-to-end Deep Learning \\ Fall 2024}
\author{John Doe - \texttt{jdoe@ur.rochester.edu}}
\date{}					

\begin{document}

\maketitle

\begin{center}
    \textbf{Deadline:} See Blackboard    
\end{center}


\section*{Instructions}

Your homework solution must be typed and prepared in \LaTeX. It must be output to PDF format. To use \LaTeX, we suggest using \url{http://overleaf.com}, which is free.

Your submission must cite any references used (including articles, books, code, websites, and personal communications).  All solutions must be written in your own words, and you must program the algorithms yourself. \textbf{If you do work with others, you must list the people you worked with.} Submit your solutions as a PDF to Blackboard. 


Your programs must be written in Python. 
If a problem requires code as a deliverable, then the code should be shown as part of the solution. One easy way to do this in \LaTeX \, is to use the verbatim environment, i.e., \textbackslash begin\{verbatim\} YOUR CODE \textbackslash end\{verbatim\}.



%%%%%%%%%%%%%%%%%%%%%%%%%%%%%%%%%%%%%%%%%%%%%

\paragraph{About Homework 1:} Homework 1 aims to acquaint you with hyperparameter tuning, network fine-tuning, WandB for Training Monitoring, and model testing. \emph{Keep in mind that network training is time-consuming, so begin early!} Copy and paste this template into an editor, e.g., \url{www.overleaf.com}, and then just type the answers in. You can use a math editor to make this easier, e.g., CodeCogs Equation Editor or MathType. \href{https://blog.writefull.com/texgpt-harness-the-power-of-chatgpt-in-overleaf/}{You may use the AI (LLM) plugin for Overleaf for help you with \LaTeX formatting.}


%CodeCogs: \url{https://www.codecogs.com/latex/eqneditor.php}

%MathType: \url{http://www.dessci.com/en/products/mathtype/}
%For MathType, you have to tell it to export as LaTex. 


\clearpage

\section*{Problem 1 - WandB for Training Monitoring}

Training neural networks involves exploring different model architectures, hyperparameters, and optimization strategies. Monitoring these choices is crucial for understanding and improving model performance. Logging experiment results during training helps to:
\begin{itemize}
    \item Gain insights into model behavior (e.g., loss, accuracy, convergence patterns).
    \item Optimize hyperparameters by evaluating their impact on stability and accuracy.
    \item Detect overfitting or underfitting and make necessary adjustments.
\end{itemize}

In this problem, you'll train ResNet-18 models for image classification on the \href{https://www.kaggle.com/datasets/tanlikesmath/the-oxfordiiit-pet-dataset?select=images}{Oxford-IIIT Pet Dataset} while exploring various hyperparameters. You’ll use Weights and Biases (W\&B) to log your experiments and refine your approach based on the results.

\subsection*{Part 1: Implementing Experiment Logging with W\&B (6 points)}

\noindent \textbf{Prepare the Dataset.} Download the dataset and split it into training, validation, and test sets as defined in \texttt{oxford\_pet\_split.csv}. Complete the dataset definition in \texttt{train.py}. During preprocessing, resize the images to $224$ as required by ResNet-18, and apply \href{https://discuss.pytorch.org/t/understanding-transform-normalize/21730}{image normalization} using \href{https://stackoverflow.com/questions/73350133/how-to-calculate-mean-and-standard-deviation-of-a-set-of-images}{statistics from the training set} or from \href{https://stackoverflow.com/questions/58151507/why-pytorch-officially-use-mean-0-485-0-456-0-406-and-std-0-229-0-224-0-2}{ImageNet}.

\noindent \textbf{Evaluating Model Performance.}
During model training, the validation set is a crucial tool to prevent overfitting. Complete \texttt{evaluate()} function in \texttt{train.py} which takes a model and a dataloader as inputs and outputs the model's accuracy score and cross-entropy loss on the dataset. 

\noindent \textbf{Integrate W\&B Logging.}
To integrate W\&B for experiment logging, follow these steps and add the necessary code to \texttt{train.py}:
\begin{enumerate}
    \item Refer to the W\&B \href{https://docs.wandb.ai/tutorials/experiments}{official tutorial} for guidance.
    \item Initialize a new run at the start of the experiment following the tutorial's code snippet. Log basic experiment \textbf{configurations}, such as total training epochs, learning rate, batch size, and scheduler usage. Ensure the run \textbf{name} is interpretable and reflects these key details.
    \item During training, log the training loss and learning rate after each mini-batch.
    \item After each epoch, log the validation loss and validation accuracy.
    \item At the end of the training, log the model's performance on the test set, including loss and accuracy scores.
\end{enumerate}

\noindent \textbf{Experiment and Analysis. }
Execute the experiment using the \textbf{default} setup. Log in to the W\&B website to inspect your implementation. 

\noindent\textbf{Deliverable:}
\begin{itemize}
    \item Screenshot(s) of the experiment configuration (under the Overview tab)
    \item Screenshot(s) of all logged charts (under the Charts tab).
    \item Are the data logged accurately in the W\&B interface? Does the experiment configuration align with your expectations?
    \item Analyze the logged charts to determine whether the training has converged.
\end{itemize}


\textbf{Answer:} \\


\subsection*{Part 2: Tuning Hyperparameters}
In this section, you'll experiment with key hyperparameters like learning rate and scheduler. For each step, change only one configuration at a time. Try not modify other hyperparameters (except batch size, which can be adjusted based on your computing resources).

\subsubsection*{2.1. Learning Rate Tuning with Sweep (5 points)}
 The learning rate is a crucial hyperparameter that significantly affects model convergence and performance.  Run the training script using \href{https://docs.wandb.ai/guides/sweeps/walkthrough}{W\&B sweep} with the following learning rates: $1e-2$, $1e-4$, and $1e-5$. Also, include the default learning rate ($1e-3$) from Part 1 in your analysis.

\noindent\textbf{Deliverable:}
\begin{itemize}
    \item Provide screenshots of logged charts showing learning rate, training loss, validation accuracy, and final test accuracy. Each chart should display results from \textbf{multiple runs} (all four learning rates in one chart). Ensure that titles and legends are clear and easy to interpret.
    \item Analyze how the learning rate impacts the training process and final performance.
    \item Code of your \href{https://docs.wandb.ai/guides/sweeps/define-sweep-configuration}{sweep configuration} that defines the search space.
\end{itemize}

\textbf{Answer:} \\

\subsubsection*{2.2. Learning Rate Scheduler (4 points)}

Learning rate schedulers dynamically adjust the learning rate during training, improving efficiency, convergence, and overall performance. In this step, you'll implement the \texttt{OneCycleLR} scheduler in the \texttt{get\_scheduler()} function within \texttt{train.py}. Compare the results to the baseline (default setting). If implemented correctly, the learning rate will initially increase and then decrease during training.

\noindent\textbf{Deliverable:}
\begin{itemize}
    \item Provide charts comparing the new setup with the baseline: learning rate, training loss, validation accuracy, and final test accuracy.
    \item Explain how the \texttt{OneCycleLR} scheduler impacts the learning rate, training process, and final performance compared to the baseline.
\end{itemize}

\textbf{Answer:} \\



\subsection*{Part 3: Scaling Learning Rate with Batch Size (5 points)}

As observed in previous parts, the choice of learning rate is crucial for effective training. As batch size increases, the effective step size in the parameter space also increases, requiring adjustments to the learning rate. In this section, you'll investigate how to scale the learning rate appropriately when the batch size changes. Read the first few paragraphs of \href{https://www.cs.princeton.edu/~smalladi/blog/2024/01/22/SDEs-ScalingRules/}{this blog post} to understand scaling rules for Adam (used in default) and SGD optimizers. Then, conduct experiments to verify these rules. First, double (or halve) the batch size without changing the learning rate and run the training script. Next, ONLY adjust the learning rate as suggested in the post. Compare these results with the default setting. Note that since the total training steps vary with batch size, you should also log the number of seen examples to create accurate charts for comparison.


\noindent\textbf{Deliverable:}
\begin{itemize}
    \item Present charts showing: training loss and validation accuracy (with the x-axis being \texttt{seen\_examples}), and final test accuracy. Ensure the legends are clear. You may apply smoothing for better visualization. 
    \item Analyze the results: do they align with the patterns discussed in the blog post?
\end{itemize}

\textbf{Answer:} \\

\subsection*{Part 4: Fine-Tuning a Pretrained Model (5 points)}
Fine-tuning leverages the knowledge of models trained on large datasets by adapting their weights to a new task. In this section, you will fine-tune a ResNet-18 model pre-trained on ImageNet using \texttt{torchvision.models.resnet18(pretrained=True)}. Modify the classification head to match the number of classes in your task, and replace the model definition in the original code. Keep the rest of the setup as default for comparison.


\noindent\textbf{Deliverable:}
\begin{itemize}
    \item Present charts showing: training loss, validation accuracy, and final test accuracy.
    \item Analyze the impact of pre-training on the model's learning process and performance.
    
\end{itemize}

\textbf{Answer:} \\



\section*{Problem 2 - Model Testing}
Unlike model evaluation, which focuses on performance metrics, model testing ensures that a model behaves as expected under specific conditions:

\noindent \textbf{Pre-Train Test:} Conducted before training, these tests identify potential issues in the model's architecture, data preprocessing, or other components, preventing wasted resources on flawed training.
 \textbf{Post-Train Test:} Performed after training, these tests evaluate the model's behavior across various scenarios to ensure it generalizes well and performs as expected in real-world situations.

In this problem, you will examine the code and model left by a former employee who displayed a lack of responsibility in his work. The code can be found in the \texttt{Problem 2} folder. The necessary predefined functions for this task are available in the \texttt{model\_testing.py} file. Follow the instructions provided in that file for detailed guidance.

\subsection*{Part 1: Pre-Train Testing}

\textbf{For each question in this part, provide clear deliverables of the following:} 1. Observations and analysis of the results; 2. Suggested approaches for addressing the detected issues (if any); 3. Code implementation.

\subsubsection*{Data Leakage Check (3 points)}
Load the training, validation, and test data sets using \texttt{get\_dataset()} function. Check for potential data leakage between these sets by directly comparing the images, as data augmentation was not applied. Since identical objects usually have different hash values in Python, consider using techniques like image hashing for this comparison.

\textbf{Answer:} \\

\subsubsection*{Model Architecture Check (2 points)}
Initialize the model using the \texttt{get\_model()} function. Verify that the model’s output shape matches the label format (hint: consider the number of classes in the dataset).

\textbf{Answer:} \\ 

\subsubsection*{Gradient Descent Validation (2 points)}
Verify that ALL the model's trainable parameters are updated after a single gradient step on a batch of data.

\textbf{Answer:} \\


\subsubsection*{Learning Rate Check (2 points)}
 Implement the learning rate range test using \href{https://github.com/davidtvs/pytorch-lr-finder#tweaked-version-from-fastaiauto}{pytorch-lr-finder}. Determine whether the learning rate is appropriately set by examining the loss-learning rate graph. Necessary components for \texttt{torch\_lr\_finder.LRFinder} are provided in \texttt{model\_testing.py}.


\textbf{Answer:} \\

\subsection*{Part 2: Post-Train Testing}

\subsubsection*{Dying ReLU Examination (4 points)}
In this section, you will examine the trained model for ``Dying ReLU." \href{https://datascience.stackexchange.com/questions/5706/what-is-the-dying-relu-problem-in-neural-networks}{Dying ReLU} occurs when a ReLU neuron outputs zero consistently and cannot differentiate between inputs. 
Load the trained model using \texttt{get\_trained\_model()} function, and the test set using \texttt{get\_test\_set()} function. Review the model's architecture, which is based on ResNet and can be found in \texttt{utils/trained\_models.py}. Then address the following:


\begin{enumerate}
    \item Identify the layer(s) where Dying ReLU might occur and explain why.
    \item Describe your approach for detecting Dying ReLU neurons. 
    \item Determine if Dying ReLU neurons are present in the trained model, and provide your code implementation..
\end{enumerate}
\textbf{Hint}: Consider how BatchNorm operation would influence the presence of dying ReLU.
    
\textbf{Answer:} \\

\subsubsection*{Model Robustness Test - Brightness (4 points)}
In this section, you will evaluate the model's robustness to changes in image brightness using a defined brightness factor.
Define a brightness factor $\lambda$, which determines the image brightness by multiplying pixel values by $\lambda$. Specifically, $\lambda = 1$ corresponds to the original image's brightness.
Load the trained model using \texttt{get\_trained\_model()} function, and the test dataset using \texttt{get\_test\_set()} function.  Investigate the model's performance across various brightness levels by adjusting $\lambda$ from $0.2$ to $1.0$ in increments of $0.2$. 

\noindent\textbf{Deliverable:}

\begin{enumerate}
    \item  Plot a curve showing how model accuracy varies with brightness levels.
    \item Analyze the relationship and discuss any trends observed.
\end{enumerate}


\textbf{Answer:} \\


\subsubsection*{Model Robustness Test - Rotation (4 points)}

Evaluate the model's robustness to changes in image rotation. Rotate the input image from $0$ to $300$ degrees in increments of $60$ degrees. 
Similarly, load the trained model using \texttt{get\_trained\_model()} function, and the test set using \texttt{get\_test\_set()} function. 

\noindent\textbf{Deliverable:}

\begin{enumerate}
    \item Plot a curve showing the relationship between rotation angles and model accuracy.
    \item Analyze the trend and discuss any observed patterns.
    \item Suggest potential improvements to enhance model robustness
\end{enumerate}



\textbf{Answer:} \\


\subsubsection*{Normalization Mismatch (2 points)}

Load the test set using the \texttt{get\_test\_set()} function. Assume that the mean and standard deviation (std) used to normalize the testing data are different from those applied to the training data.

\noindent\textbf{Deliverable:}

\begin{enumerate} \item Calculate and report the mean and std of the images in the loaded test set (\href{https://stackoverflow.com/questions/73350133/how-to-calculate-mean-and-standard-deviation-of-a-set-of-images}{tutorial}). Compare these values with the expected mean and std after proper normalization.

\item Discuss one potential impact of this incorrect normalization on the model's performance or predictions. 

\end{enumerate}
\textbf{Answer:} 


\end{document}  